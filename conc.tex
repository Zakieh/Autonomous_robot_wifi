%\vspace{-5pt}
\section{Conclusion}
\label{sec:conc}
%\vspace{\vertspcposthead}
%In this paper, we proposed three variants to the RGBD SLAM algorithm, each of which modified the algorithm to use RSSI information from commodity-off-the-shelf Wi-Fi cards to disambiguate different locations in an indoor office environment that is repetitive or feature-less. We demostrated the quality of our method by comparing the results produced against the results produced by existing RGBD SLAM algorithm. Results from our experiments show that our variants produce trajectories that are up to 18 times more accurate than RGBD SLAM and have the added benefit of having the computation time reduced by up to 70\%.
%\vspace{-15pt}
%\vspace{-35pt}
%\vspace{-35pt}
In this work, we proposed a general approach to incorporate Wi-Fi sensing into visual SLAM algorithms. To demonstrate, 
we augment three recent SLAM algorithms to show improved mapping/localization accuracy as well as speed-up in operation. 
We demonstrated this functionality on data collected from four university buildings, which are representative spaces of modern urban environments.

We also compared our proposed approach to recently proposed Wi-Fi augmented FABMAP and showed a comparable if not better performance.
This comparison also confirms the generality of our approach unlike Wi-Fi augmented FABMAP which is only designed for visual FABMAP.

In the future, we hope to demonstrate the utility of Wi-Fi sensing for sustained long-term use in an urban space. 
While Wi-Fi signal strength used for this work is a useful measure, novel wireless sensing technologies such as 60 GHz sensing can be used to further improve the overall accuracy as well as the computational complexity of SLAM algorithms in the future as well.
%more detailed properties such as Channel State Information (CSI) can be obtained using modern Wi-Fi cards. These would greatly enhance Wi-Fi sensing ability and its corresponding use for multi-robot applications. We will explore using CSI for SLAM in the future.
